% !TEX encoding = UTF-8
% !TEX TS-program = pdflatex
% !TEX root = ../tesi.tex

%**************************************************************
\chapter{Progettazione e codifica}
\label{cap:progettazione}
In questo capitolo verrà tratta il secondo periodo del tirocinio: la progettazione e la codifica. Questa parte, avvenuta dopo l'Analisi dei Requisiti, è da ritenersi importante per progetto in quanto ha occupato gran parte del tempo lavorativo. Il primo passo è stato eseguire la progettata dell'architettura, in modo che i servizi necessari fossero correttamente configurati prima del loro effettivo utilizzo. Successivamente è stata eseguita la stesura del codice per la realizzazione della Skill. Nei paragrafi successivi seguiranno porzioni di codice significativo per la loro importanza e per il funzionamento di determinate funzionalità e/o servizi. Importante ricordare che il codice implementato e mostrato sarà in Node.js come quanto detto nel nel paragrafo \hyperref[nodejs]{1.4.1}. 
%**************************************************************
\section{Architettura}
Come riportato nel paragrafo \hyperref[serivizi_aws]{1.4.3}, il progetto si basa interamente sui servizi offerti dall'ecosistema Amazon. Di conseguenza è risultato facile ed immediato impostare i servizi dell'architettura in modo da rendere il tutto efficacie ed efficiente. In questo paragrafo si andrà a comprendere cosa offre ogni singolo servizio di AWS e come esso è stato impostato per il suo corretto funzionamento nella Skill.
\section{Configurazione servizi Amazon Web Service}
\subsection{AWS SES}
Amazon SES\footnote{\href{AWS SES. URL: https://aws.amazon.com/it/ses/}{https://aws.amazon.com/it/ses/}} (Simple Email Service) è un servizio di invio e-mail basato sul cloud messo a disposizione da Amazon per gli sviluppatori di applicazioni. Tale servizio è da considerarsi vantaggioso in quanto è affidabile e a costo ridotto, utile per qualunque tipo di azienda ed ideale per la realizzazione del progetto.
\newpage
\noindent Per utilizzare AWS SES è necessario impostare il servizio visitando la pagina\footnote{\href{AWS SES. URL: https://aws.amazon.com/it/ses/}{https://aws.amazon.com/it/ses/}} dedicata ed eseguire le istruzioni riportate:\\[0.5cm]
\begin{minipage}{0.47\textwidth}
	\begin{figure}[H]
		\includegraphics[width=6cm]{immagini/ses.png}
		\caption{\label{fig:icona_aws_ses}Icona AWS SES}
	\end{figure}
\end{minipage}
\begin{minipage}{0.5\textwidth}
	\begin{itemize}
		\item Eseguire l'accesso con le proprie credenziali di account AWS
    	\item Cliccare \texttt{Email Addresses }sul pannello a sinistra
    	\item Cliccare \texttt{Verify a New Email Address}
    	\item Inserire l'indirizzo e-mail da verificare che verrà utilizzato per inviare le notifiche
    	\item Confermare la verifica cliccando sul URL contenuto nella e-mail ricevuta.
	\end{itemize}
\end{minipage}
\\[0.5cm]
Il procedimento descritto sopra non fa altro che verificare l'indirizzo e-mail con il quale si andrà ad inviare messaggi di posta elettronica tramite la Skill. A questo punto è possibile mandare messaggi e notifiche tramite mail utilizzando l'indirizzo verificato prima.
\subsection{AWS S3}
\subsection{AWS IAM}
\subsubsection{AWS Cloud Watch}
\subsubsection{AWS DynamoDB}
\subsection{AWS Lambda ed Alexa Developer Console}

\section{Configurazione Google Calendar}

\section{Configurazione Slack}

\section{Codifica}
\subsection{Organizzazione del codice}
\subsection{Invio notifiche}
\subsubsection{Slack}
\subsubsection{E-mail}
\subsection{Lettura del calendario}
\subsection{Lettura del Database}
\subsection{Cambio di contesto}
