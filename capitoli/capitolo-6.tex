% !TEX encoding = UTF-8
% !TEX TS-program = pdflatex
% !TEX root = ../tesi.tex

%**************************************************************
\chapter{Valutazione finale}
\label{cap:valutazione_finale}
%**************************************************************
In questo ultimo capitolo verranno fatte le analisi dei risultati otteniti, dei dati e delle informazioni raccolte ed emerse durante il tirocinio e una valutazione personale di come è stata vissuta tale esperienza.
\\[0.5cm]
Il progetto di stage aveva come obiettivo la realizzazione di una Skill, pensata per Amazon Echo Show, per l'accoglienza di visitatori utilizzando le funzionalità e i servizi offerti da Amazon Web Services.
Lo scopo principale del tirocinio era la sperimentazione con le tecnologie Node.js ed Alexa Presentation Language per realizzare un prototipo di prodotto su cui individuare le potenzialità i limiti limitazioni, confrontandole anche con le tecnologie simili come Google Assistant. In tale contesto, a inizio stage assieme al tutor aziendale sono stati definito gli obiettivi minimi e massimi da portare a termine nell'arco delle 300-320 ore a disposizione. Gli obiettivi minimi racchiudono tutte le attività incentrate sullo studio ,sull'utilizzo delle tecnologie AWS e l'integrazione di servizi terzi finalizzati al mio stage. Gli obiettivi massimi, invece, riguardavano il soddisfacimento dei requisiti desiderabili quale la registrazione della durata degli appuntamenti determinati dall'inizio e dalla fine dell'accoglienza fisica fatta dal personale.
% - Risultato ottenuto
% - Analisi critica del prodotto e del lavoro di stage in generale
% - Il prodotto e utilizzato?
% - Valutazione degli strumenti utilizzati
% - Possibili punti di insoddisfazione e relativi miglioramenti
% - Possibili estensioni
\section{Risultati ottenuti}
A valle del lavoro svolto e dell'impegno dedicato, è possibile affermare che il risultato ottenuto dal progetto Concierge Croccante è da considerarsi soddisfacente rispetto alle attese. Questo motivato dal fatto che al termine del periodo di tirocinio gli obbiettivi prefissati come obbligatori sono stati considerarsi completamente soddisfatti. In particolari, facendo riferimento all'identificazione dei requisiti riportata al punto \hyperref[indentificazione-requisiti]{2.3.1}, sono ritenuti soddisfatti i seguenti:
\begin{center}
	\centering
	\renewcommand{\arraystretch}{1.5}
	\rowcolors{3}{tableLight}{}
	\begin{longtable}{  p{2.5cm} p{2.5cm} }
		\rowcolor{tableHead}
		\textbf{\textcolor{white}{Identificativo}} & \textbf{\textcolor{white}{Soddisfatto}} \\
		\endhead  
		RO1 & Si \\
		RO2 & Si \\ 
		RO3 & Si \\
		RO4 & Si \\
		RO5 & Si \\
		RO6 & Si \\
		RO7 & Si \\
		RO8 & Si \\
		RO9 & Si \\
		RO10 & Si \\
		RO11 & Si \\
	    RO12 & Si \\
	    RO13 & Si \\
	    RO14 & Si \\
	    RO15 & Si \\
	    RO16 & Si \\
	    RO17 & Si \\
		RO18 & Si \\
		RO19 & Si \\
		RD1 & Si \\
		RD2 & No \\
		RD3 & No \\
		RD4 & No \\
		\rowcolor{white}
		\caption{Tabella requisti soddisfatti}
	\end{longtable}
\end{center}
Inoltre, grazie ad una buona integrazione con il personale al di fuori del tutor aziendale, è stato possibile collaborare con il team UI/UX per una maggiore completezza e presentazione della Skill Concierge Croccante.
Un grande apprezzamento come risultato dello stage è aver conosciuto ed approfondito l'utilizzo delle API dei servizi Google Calendar e Slack integrati su ambienti cloud computing.

\section{Analisi critica del prodotto}
Analizzando criticamente il prodotto si può affermare che la Skill è stata sviluppata con una buona base conversazionale semplice e non eccessivamente lunga da svolgere, fattore rilevante visto il contesto degli assistenti vocali. Altro valore aggiunto è il fatto di aver realizzato un prodotto con funzionalità ben definite, così da non causare smarrimento all'utente durante l'utilizzo.
\\[0.5cm]
Osservando la tabella riportata sopra il requisito:\\
\begin{itemize}
    \item RD2 non è stato soddisfatto a causa del poco tempo rimasto a disposizione durante il periodo di progettazione e codifica. Era stato però pensato di implementare questa funzionalità utilizzando un secondo calendario dove inserire i periodi di assenza del dipendente;
    \item RD3 non è stato possibile completato in quanto non è stata trovata alcuna soluzione automatica in grado di soddisfare tale requisito senza obbligare l'utente ad indicare l'inizio dell'appuntamento alla Skill;
	\item RD4 non è stato possibile completato in quanto non è stata trovata alcuna soluzione automatica in grado di soddisfare tale requisito senza obbligare l'utente ad indicare il termine dell'appuntamento alla Skill.
\end{itemize}

\section{Analisi critica del lavoro}
Nel corso dello stage ho avuto la possibilità di mettere alla prova le mie competenze di organizzazione del lavoro e pianificazione delle attività per la prima volta al di fuori dell'ambito universitario, competenze acquisite per la maggior parte dal corso di Ingegneria del Software\footnote{SWE. URL: \href{https://www.math.unipd.it/~tullio/IS-1/2018/}{https://www.math.unipd.it/~tullio/IS-1/2018/}}. Nonostante io sia stato seguito ed aiutato dal mio tutor aziendale, la gestione temporale del progetto è stata per la maggior parte affidata a me. L'interazione con il tutor in questo contesto è stata principalmente espressa da incontri per ricevere feedback sull'organizzazione delle attività e sullo stato dei lavori. Questo mi ha permesso di maturare in maniera autonoma la priorità delle varie fasi, migliorando così le mie capacità organizzative grazie all'esperienza diretta. Altra soddisfazione ottenuta è data dall'esperienza acquisita utilizzando i servizi offerti dall'ecosistema Amazon Web Service con il quale è stato progettato il prodotto finale, molto apprezzato visto il numeroso ambito d'uso. Complessivamente sono molto soddisfatto dell'esperienza di stage svolta presso Crispy Bacon spa: oltre all'ottimo ambiente di lavoro e del personale, riscontrato in azienda, ho avuto modo di accrescere la mia figura professionale. Alla luce delle conoscenze e competenze acquisite durante il tirocinio posso affermare con sicurezza che le mie aspettative personali sono state completamente corrisposte.

\section{Valutazione degli strumenti utilizzati}
\section{Possibili miglioramenti}
\section{Possibili estensioni}