% !TEX encoding = UTF-8
% !TEX TS-program = pdflatex
% !TEX root = ../tesi.tex

%**************************************************************
\chapter{Introduzione}
\label{cap:introduzione}
%**************************************************************
Introduzione al contesto applicativo.\\
\noindent Esempio di utilizzo di un termine nel glossario \\
\gls{api}. \\

\noindent Esempio di citazione in linea \\
\cite{site:agile-manifesto}. \\

\noindent Esempio di citazione nel pie' di pagina \\
citazione\footcite{womak:lean-thinking} \\

%**************************************************************
\section{Azienda ospitante}
Le aziende ospitano uno stage
\subsection{Crispy Bacon}
Descrizione dell'azienda.
%**************************************************************
\section{Progetto di stage}
\subsection{Alexa}
che cosa è Alexa
\subsection{Amazon AWS}
Che cosa è Amazon AWS
\subsection{L'idea}
Da dove nasce l'idea
\subsection{Interesse aziendale}
Che interessi ha l'azienda
\subsection{Principali problematiche}
Problemi che ci sono nel realizzare tale proposta
\subsection{Soluzione proposta}
Ecco Concierge Croccante

%**************************************************************
\section{Organizzazione del testo}
Il documento presenta la seguente struttura:
\begin{itemize}
    \item Capitolo 1: Introduzione: comprende la descrizione generale dell’azienda ospitante e del relativo metodo di lavoro, una contestualizzazione ed illus- trazione del progetto di stage, la presentazione della struttura del documento e delle convenzioni stilistiche e sintattiche adottate;
    \item Capitolo 2: Analisi dei Requisiti: si tratta di una descrizione dettagliata della fase di analisi dei requisiti portata a termine dal candidato;
    \item Capitolo ??: Ricerca: viene descritta l’attivit`a di ricerca effettuata al fine di raccogliere informazioni sul problema in esame;
\end{itemize}




