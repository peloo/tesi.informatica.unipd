% Tesi di laure triennale informatica 
% Matteo pellanda A.A. 2018/2019

% I seguenti commenti speciali impostano:
% 1. 
% 2. PDFLaTeX come motore di composizione;
% 3. tesi.tex come documento principale;
% 4. il controllo ortografico italiano per l'editor.

% !TEX encoding = UTF-8
% !TEX TS-program = pdflatex
% !TEX root = tesi.tex
% !TEX spellcheck = it-IT

\documentclass[10pt,                    % corpo del font principale
               			a4paper,                 % carta A4
               			twoside,                 % impagina per fronte-retro
              			openright,               % inizio capitoli a destra
               			english,                 
               			italian,                 
               			]{book}    

%**************************************************************
% Importazione package
%************************************************************** 

%\usepackage{amsmath,amssymb,amsthm}    % matematica

\usepackage[T1]{fontenc}                % codifica dei font:
                                        % NOTA BENE! richiede una distribuzione *completa* di LaTeX

\usepackage[utf8]{inputenc}             % codifica di input; anche [latin1] va bene
                                        % NOTA BENE! va accordata con le preferenze dell'editor

\usepackage[english, italian]{babel}    % per scrivere in italiano e in inglese;
                                        % l'ultima lingua (l'italiano) risulta predefinita

\usepackage{bookmark}                   % segnalibri

\usepackage{caption}                    % didascalie

\usepackage{chngpage,calc}              % centra il frontespizio

\usepackage{csquotes}                   % gestisce automaticamente i caratteri (")

\usepackage{emptypage}                  % pagine vuote senza testatina e piede di pagina

\usepackage{epigraph}			% per epigrafi

\usepackage{eurosym}                    % simbolo dell'euro

%\usepackage{indentfirst}               % rientra il primo paragrafo di ogni sezione

\usepackage{graphicx}                   % immagini

\usepackage{hyperref}                   % collegamenti ipertestuali

\usepackage[binding=5mm]{layaureo}      % margini ottimizzati per l'A4; rilegatura di 5 mm

% \usepackage{listings}                   % codici

\usepackage{microtype}                  % microtipografia

\usepackage{mparhack,fixltx2e,relsize}  % finezze tipografiche

\usepackage{nameref}                    % visualizza nome dei riferimenti                                      

\usepackage[font=small]{quoting}        % citazioni

\usepackage{subfig}                     % sottofigure, sottotabelle

\usepackage[italian]{varioref}          % riferimenti completi della pagina


\usepackage[table,dvipsnames]{xcolor}         % colori

\usepackage{booktabs}                   % tabelle                                       
\usepackage{tabularx}                   % tabelle di larghezza prefissata                                    
\usepackage{longtable}                  % tabelle su più pagine                                        
\usepackage{ltxtable}                   % tabelle su più pagine e adattabili in larghezza

\usepackage[toc, acronym]{glossaries}   % glossario
                                        % per includerlo nel documento bisogna:
                                        % 1. compilare una prima volta tesi.tex;
                                        % 2. eseguire: makeindex -s tesi.ist -t tesi.glg -o tesi.gls tesi.glo
                                        % 3. eseguire: makeindex -s tesi.ist -t tesi.alg -o tesi.acr tesi.acn
                                        % 4. compilare due volte tesi.tex.

\usepackage[backend=biber,style=verbose-ibid,hyperref,backref]{biblatex}
                                        % eccellente pacchetto per la bibliografia; 
                                        % produce uno stile di citazione autore-anno; 
                                        % lo stile "numeric-comp" produce riferimenti numerici
                                        % per includerlo nel documento bisogna:
                                        % 1. compilare una prima volta tesi.tex;
                                        % 2. eseguire: biber tesi
                                        % 3. compilare ancora tesi.tex.

% -------------------------------------------------------------------
% Package tabella
% -------------------------------------------------------------------
\usepackage{multirow}
\usepackage{longtable}
\usepackage{xcolor}
\definecolor{tableHead}{RGB}{158,9,46}
\definecolor{tableLight}{RGB}{255, 247, 234}
% -------------------------------------------------------------------

\usepackage{float}

% -------------------------------------------------------------------
% Package codice
% -------------------------------------------------------------------
\usepackage{listings}
\definecolor{lightgray}{rgb}{.9,.9,.9}
\definecolor{purple}{rgb}{0.0, 0.5, 0.0}
\lstdefinelanguage{JavaScript}{
	keywords={typeof, new, true, false, catch, function, return, null, catch, switch, var, if, in, while, do, else, case, break},
	keywordstyle=\color{blue}\bfseries,
	ndkeywords={class, export, boolean, throw, implements, import, this, console, module},
	ndkeywordstyle=\color{blue}\bfseries,
	identifierstyle=\color{black},
	sensitive=false,
	comment=[l]{//},
	morecomment=[s]{/*}{*/},
	commentstyle=\color{purple}\ttfamily,
	stringstyle=\color{tableHead}\ttfamily,
	morestring=[b]',
	morestring=[b]"
}

% -------------------------------------------------------------------

\input{tesi-config}                     % file con le impostazioni personali

\begin{document}
%**************************************************************
% Materiale iniziale
%**************************************************************
\frontmatter
\input{inizio-fine/frontespizio}
\input{inizio-fine/colophon}
% \input{inizio-fine/dedica}
% !TEX encoding = UTF-8
% !TEX TS-program = pdflatex
% !TEX root = ../tesi.tex

%**************************************************************
% Sommario
%**************************************************************
\cleardoublepage
\phantomsection
\pdfbookmark{Sommario}{Sommario}
\begingroup
\let\clearpage\relax
\let\cleardoublepage\relax
\let\cleardoublepage\relax

\chapter*{Sommario}

Il presente documento descrive il lavoro svolto durante il periodo di stage, della durata di circa trecento ore, dal laureando Matteo Pellanda presso l'azienda Crispy Bacon Srl. Lo stage è stato svolto al termine del percorso di studi della laurea triennale in Informatica, ed ha avuto la durata di trecentoventi ore.\\[0.4cm]
L'obbiettivo era di realizzare una Skill per l'assistente vocale Amazon Alexa, utilizzando il linguaggio NodeJS. Prima della sua realizzazione è stato redatto un documento di analisi del prodotto e alla fine una documentazione sulle tecnologie software utilizzate.\\[0.4cm]
Il presente documento ha lo scopo di illustrare il contesto aziendale dove è stato svolto lo stage, le attività svolte durante esso, ed infine una valutazione sul lavoro effettuato.
La Skill realizzata ha nome Concierge Crocante, che nel documento potrà essere abbreviato con l'acronimo C.C.  

%\vfill
%
%\selectlanguage{english}
%\pdfbookmark{Abstract}{Abstract}
%\chapter*{Abstract}
%
%\selectlanguage{italian}

\endgroup			

\vfill


% !TEX encoding = UTF-8
% !TEX TS-program = pdflatex
% !TEX root = ../tesi.tex

%**************************************************************
% Ringraziamenti
%**************************************************************
\cleardoublepage
\phantomsection
\pdfbookmark{Ringraziamenti}{ringraziamenti}

\begin{flushright}{
	\slshape    
	``Life is really simple, but we insist on making it complicated''} \\ 
	\medskip
    --- Confucius
\end{flushright}


\bigskip

\begingroup
\let\clearpage\relax
\let\cleardoublepage\relax
\let\cleardoublepage\relax

\chapter*{Ringraziamenti}

\noindent \textit{Innanzitutto, vorrei esprimere la mia gratitudine al Prof. Paolo Baldan, relatore della mia tesi, per l'aiuto e il sostegno fornitomi durante la stesura del lavoro.}\\

\noindent \textit{Desidero ringraziare con affetto i miei genitori per il sostegno, il grande aiuto ricevuto e per essermi stati vicini in ogni momento durante gli anni di studio.}\\

\noindent \textit{Inoltre desiderio ringraziare i miei amici e compagni di corso per tutti i bellissimi anni passati insieme e le mille avventure vissute.}\\
\bigskip

\noindent\textit{\myLocation, \myTime}
\hfill \myName

\endgroup


\input{inizio-fine/indici}
\cleardoublepage

%**************************************************************
% Materiale principale
%**************************************************************

% * INTRODUZIONE
% - Breve descrizione del progetto
% - Principali problematiche
% - Soluzione scelta
% - Strumenti utilizzati
% - Descrizione del prodotto ottenuto
% - Sruttura del resto della relazione

% * ANALISI DEI REQUISITI

% * PROGETTAZIONE

% * REALIZZAZIONE E TESTING

% * CONCLUSIONI

% - Risultato ottenuto
% - Analisi critica del prodotto e del lavoro di stage in generale
% - Il prodotto e utilizzato?
% - Valutazione degli strumenti utilizzati
% - Possibili punti di insoddisfazione e relativi miglioramenti
% - Possibili estensioni

\mainmatter
% !TEX encoding = UTF-8
% !TEX TS-program = pdflatex
% !TEX root = ../tesi.tex

%**************************************************************
\chapter{Introduzione}
\label{cap:introduzione}
%**************************************************************
Introduzione al contesto applicativo.\\
\noindent Esempio di utilizzo di un termine nel glossario \\
\gls{api}. \\

\noindent Esempio di citazione in linea \\
\cite{site:agile-manifesto}. \\

\noindent Esempio di citazione nel pie' di pagina \\
citazione\footcite{womak:lean-thinking} \\

%**************************************************************
\section{Azienda ospitante}
Le aziende ospitano uno stage
\subsection{Crispy Bacon}
Descrizione dell'azienda.
%**************************************************************
\section{Progetto di stage}
\subsection{Alexa}
che cosa è Alexa
\subsection{Amazon AWS}
Che cosa è Amazon AWS
\subsection{L'idea}
Da dove nasce l'idea
\subsection{Interesse aziendale}
Che interessi ha l'azienda
\subsection{Principali problematiche}
Problemi che ci sono nel realizzare tale proposta
\subsection{Soluzione proposta}
Ecco Concierge Croccante

%**************************************************************
\section{Organizzazione del testo}
Il documento presenta la seguente struttura:
\begin{itemize}
    \item Capitolo 1: Introduzione: comprende la descrizione generale dell’azienda ospitante e del relativo metodo di lavoro, una contestualizzazione ed illus- trazione del progetto di stage, la presentazione della struttura del documento e delle convenzioni stilistiche e sintattiche adottate;
    \item Capitolo 2: Analisi dei Requisiti: si tratta di una descrizione dettagliata della fase di analisi dei requisiti portata a termine dal candidato;
    \item Capitolo ??: Ricerca: viene descritta l’attivit`a di ricerca effettuata al fine di raccogliere informazioni sul problema in esame;
\end{itemize}




             
% !TEX encoding = UTF-8
% !TEX TS-program = pdflatex
% !TEX root = ../tesi.tex

%**************************************************************
\chapter{Analisi dei Requisiti}
\label{cap:processi-metodologie}
%**************************************************************
Durante il primo periodo di tirocinio è stata svolta la fase di Analisi dei Requisiti, necessaria alla comprensione del dominio e al soddisfacimento della richiesta. Inizialmente è stato effettuato un incontro con il tutor aziendale con il quale sono stati rilevati i requisti obbligatori richiesti dall'azienda da cui è stato possibile identificare un insieme di funzionalità necessarie alla Skill. Successivamente è stato realizzato il documento contenente tutti gli studi di analisi che hanno favorito una buona e corretta progettazione del prodotto. Dalla raccolta dei dati e dalle analisi fatte è stata elaborata una visione ad alto livello del sistema e dei rispettivi casi d’uso. Sono stati inoltre identificati quei punti critici in grado di determinare, in larga misura, la forma finale del prodotto.
La fase di analisi si è infine conclusa con lo studio della VUI (Voice User Interface), e della GUI (Graphical User Interface), che caratterizzano l'esperienza d'uso del prodotto finale. 

%**************************************************************
\section{Obbiettivo}
L’obiettivo del progetto di stage è la realizzazione di una Skill per l’assistente vocale Alexa che sia in grado di accogliere clienti, postini e corrieri all'ingresso degli uffici e notificare in maniera automatica la persona interessata nell'arrivo del visitatore. La Skill è concepita per essere installata sui dispositivi in commercio da Amazon, in particolare sul dispositivo Echo Show 2018. L’obiettivo è quindi quello di realizzare un concierge virtuale che accolga il visitatore e riceva da esso informazioni per mezzo di una conversazione e l’uso di messaggi attraverso lo schermo touch integrato nel dispositivo. In fine tali dati elaborati dai processi di controllo per inviare notifiche al personale.

\subsection{Cliente finale}
Il cliente finale a cui è destinato il prodotto è l’azienda Crispy Bacon Srl, la quale necessita, per ovvi motivi, di un sistema automatizzato che svolga il ruolo di Concierge all'entrata degli uffici. L’azienda desidera, con questo prodotto, realizzare uno prototipo da poter rendere spendibile tale servizio proponendolo come pacchetto preconfezionato o da personalizzare al fine di venderlo a clienti terzi.

\section{Attori coinvolti}
Dall'analisi fatta sono emersi i seguenti attori, intesi come persone coinvolte nell'utilizzo della Skill:
\begin{itemize}
    \item Visitatore
        \begin{itemize}
            \item Persona avente appuntamento
            \item Persona non avente appuntamento
            \item Postino
            \item Corriere
        \end{itemize}
    \item Personale di Crispy Bacon
\end{itemize}

\section{Requisiti richiesti}
I requisiti emersi e richiesti dall'azienda sono stati elaborati e analizzati. Tali requisiti sono interpretabili in tre diversi modi, tutti sotto il presupposto che questi siano in qualche modo una necessità.
\begin{itemize}
    \item Requisito utente: dal punto di vista dell'utente, è una capacità necessaria per risolvere un problema o raggiungere un obbiettivo.
    \item Requisito software: dal punto di vista della soluzione, è una capacità che deve essere posseduta dal sistema per adempiere all'obiettivo.
    \item Dal punto di vista della documentazione, come una descrizione documentata di una capacità interpretata come un requisito utente o software.
\end{itemize}
\subsubsection{Identificazione dei Requisiti}
Per identificare i requisti viene utilizzata la seguente notazione tabellare con un codice che lo identifica e la corrispettiva descrizione affianco.
\begin{center}
    R.x.y
\end{center}
\begin{itemize}
    \item R: identifica il requisito
    \item x: identifica l'importanza di tale requisito che può essere
        \begin{itemize}
            \item O obbligatorio
            \item D desiderabile
        \end{itemize}
    \item y: identifica un valore numerico progressivo a partire da 1
\end{itemize}
In merito allo studio di analisi fatto all'inizio del periodo di tirocinio sono emersi i seguenti requisti, \textbf{considerati obbligatori} per il soddisfacimento del risultato atteso dal prodotto finale.
\begin{center}
	\centering
	\renewcommand{\arraystretch}{1.5}
	\rowcolors{3}{tableLight}{}
	\begin{longtable}{  p{2.5cm} p{9.8cm} }
		\rowcolor{tableHead}
		\textbf{\textcolor{white}{Identificativo}} & \textbf{\textcolor{white}{Requisito}} \\
		\endhead  
		RO1 & La Skill al momento dal lancio deve presentare un messaggio di benvenuto (VUI) \\
		RO2 & La Skill al momento dal lancio deve presentare un messaggio di benvenuto (GUI) \\
		RO3 & La Skill al momento dal lancio deve presentare un elenco essenziale e sintetico delle azioni da fare (VUI) \\ 
		RO4 & La Skill al momento dal lancio deve presentare un elenco essenziale e sintetico delle azioni da fare  (GUI) \\
		RO5 & La Skill deve poter ricevere le informazioni necessarie dal visitatore per mezzo di una conversazione impostata (VUI)  \\
		RO6 & La Skill deve poter riportare le informazioni ottenute dal visitatore e riportarle nello schermo del dispositivo (GUI) \\
		RO7\label{RO7} & La Skill deve poter utilizzare e interrogare il servizio di calendarizzazione con le informazioni ottenute dal visitatore al fine di notificare l'interessato della visita.\\
		RO8\label{RO8} & La Skill deve notificare o inviare una notifica all'interessato della visita.\\
		RO9 & La Skill deve riportare sullo schermo del dispositivo delle funzioni disponibili in base alle risposte ricevute dal dialogo fatto con il visitatore (GUI - VUI)\\
		\rowcolor{white}
		\caption{Tabella tracciamento requisiti obbligatori}
	\end{longtable}
\end{center}
Infine nell'analisi sono stati individuati anche i \textbf{requisiti desiderabili}, considerati non strettamente necessari ma di valore aggiunto al prodotto atteso.
\begin{center}
	\centering
	\renewcommand{\arraystretch}{1.5}
	\rowcolors{3}{tableLight}{}
	\begin{longtable}{  p{2.5cm} p{9.8cm} }
		\rowcolor{tableHead}
		\textbf{\textcolor{white}{Identificativo}} & \textbf{\textcolor{white}{Requisito}} \\
		\endhead  
		RD1 & La Skill una volta verificata la presenza del visitatore informa quest'ultimo se l'interessato risulta non reperibile se assente \\
		RD2 & La Skill deve poter registrare il momento in cui il visitatore inizia l'incontro con la persona cercata nel caso di un appuntamento \\
		RD3 & La Skill deve poter registrare il momento in cui il visitatore termina l'incontro con la persona cercata nel caso di un appuntamento \\
		\rowcolor{white}
		\caption{Tabella tracciamento requisiti desiderabili}
	\end{longtable}
\end{center}
\section{Casi d'uso}
Dalle analisi fatte e dai dati raccolti sono stati studiati i requisiti funzionali, ovvero i casi d'uso del prodotto Concierge Crocante, che permettono di descrivere interazioni tra gli utenti, tra il sistema e come quest'ultimo deve essere utilizzato. Durante lo stage sono stati così esaminate le sequenze di passi che descrivono interazioni e la rappresentazione di possibilità, che hanno in comune uno scopo finale per un utente (attore).\\[0.4cm]
Gli attori emersi nell'analisi dei casi d’uso svolgono il ruolo dell'utente nell'interazione con la Skill per raggiungere l’obiettivo prefissato. In questa analisi sono stati individuati gli attori:
    \begin{itemize}
        \item Utente generico, che può essere generalizzato in:
            \begin{itemize}
                \item Visitatore
                    \begin{itemize}
                        \item avente appuntamento
                        \item non avente appuntamento
                        \item postino
                        \item corriere
                    \end{itemize}
                \item Persona cercata
        \end{itemize}
    \end{itemize}
Di seguito viene riportato lo schema degli attori individuati utilizzando lo standard UML 2.0
\begin{figure}[H] 
    \centering 
    \includegraphics[width=0.9\columnwidth]{immagini/attori.png}
    \caption{\label{fig:attori}Utenti del sistema}
\end{figure}
\subsection{Casi d'uso - Visitatore}
Durante il periodo di tirocinio, la fase di analisi dei casi d'uso è da considerarsi divisa in due parti simili fra loro: quella dedicata al visitatore, ovvero colui che si presenta negli uffici dell'azienda avente un appuntamento registrato in calendario, o che semplicemente si presenta senza preavviso, e il servizio di consegna pacchi svolto dal postino o dal corriere. In entrambe le parti gli attori avvieranno la Skill installata nel dispositivo Amazon Echo Show e seguiranno le istruzioni riportate a voce dall'assistente vocale oppure mostrate a video dal display. In questa sezione viene riporta una rappresentazione grafica dei casi d'uso dell'utente identificato come "Visitatore avente appuntamento", che descrive le interazioni che l'attore svolge con il sistema.  
\begin{figure}[H] 
    \centering 
    \includegraphics[width=1\columnwidth]{immagini/casi_duso1.png}
    \caption{\label{fig:attori}Casi d'uso visitatore avente appuntamento}
\end{figure}
Dall'elaborazione di tale analisi sono quindi emersi i seguenti casi d'uso riportati:
\begin{center}
	\centering
	\renewcommand{\arraystretch}{1.5}
	\rowcolors{3}{tableLight}{}
	\begin{longtable}{  p{2.5cm} p{9.8cm} }
		\rowcolor{tableHead}
		\textbf{\textcolor{white}{Identificativo}} & \textbf{\textcolor{white}{Descrizione}} \\
		\endhead  
		
		UC1 &  \textit{Attori}: persona con o senza appuntamento \newline \textit{Scopo}: l'utente può avviare Concierge Croccante \newline \textit{Pre-condizione}: il dispositivo Amazon deve aver avviato \mbox{l'assistente} Alexa \newline \textit{Post-condizione}: l'utente ha avviato Concierge Croccante \\
		
		UC2 &  \textit{Attori}: persona \newline \textit{Scopo}: l'utente riceve un messaggio di benvenuto sul display del dispositivo \newline \textit{Pre-condizione}: la Skill Concierge Croccante deve essere stata avviata \newline \textit{Post-condizione}: l'utente riceve un messaggio di benvenuto sul display del dispositivo \\
		
		UC3 &  \textit{Attori}: persona \newline \textit{Scopo}: l'utente riceve un messaggio vocale di benvenuto dalla Skill \mbox{Concierge} Croccante \newline \textit{Pre-condizione}: la Skill Concierge Croccante deve essere stata avviata \newline \textit{Post-condizione}: l'utente riceve un messaggio vocale dalla Skill Concierge Croccante\\
		
		UC4 &  \textit{Attori}: persona \newline \textit{Scopo}: l'utente visualizza l'elenco sintetico ed essenziale di azioni sul \mbox{display} del dispositivo \newline \textit{Pre-condizione}: la Skill Concierge Croccante deve essere stata avviata \newline \textit{Post-condizione}: l'utente visualizza l'elenco di azioni sul display del dispositivo \\
		
		UC5 &  \textit{Attori}: persona \newline \textit{Scopo}: viene esposto all'utente l'elenco sintetico ed essenziale di azioni \newline \textit{Pre-condizione}: la Skill Concierge Croccante deve essere stata avviata \newline \textit{Post-condizione}: viene esposto all'utente l'elenco di azioni disponibili da Concierge Croccante \\
		
		UC6 &  \textit{Attori}: persona con appuntamento \newline \textit{Scopo}: l'utente può annunciarsi per essere accolto dalla persona cercata \newline \textit{Pre-condizione}: la Skill Concierge Croccante deve essere stata avviata \newline \textit{Post-condizione}: l'utente si annuncia \\
		
		UC7 &  \textit{Attori}: persona con appuntamento \newline \textit{Scopo}: viene visualizzato un messaggio, su display e vocale, di errore specifico per l'eccezione riscontrata \newline \textit{Pre-condizione}: la Skill Concierge Croccante deve essere stata avviata \newline \textit{Post-condizione}: l'utente riceve un messaggio su display e vocale di errore\\
		
		UC8 &  \textit{Attori}: persona \newline \textit{Scopo}: l'utente può annunciarsi per essere accolto dalla persona cercata \newline \textit{Pre-condizione}: la Skill Concierge Croccante deve essere stata avviata \newline \textit{Post-condizione}: l'utente si annuncia \\
		
		UC9 &  \textit{Attori}: persona \newline \textit{Scopo}: viene visualizzato un messaggio, su display e vocale, di errore specifico per l'eccezione riscontrata \newline \textit{Pre-condizione}: la Skill Concierge Croccante deve essere stata avviata \newline \textit{Post-condizione}: l'utente riceve un messaggio su display e vocale di errore\\
		
		\rowcolor{white}
		\caption{\label{tab:UC_persona}Tabella casi d'uso persona con appuntamento e non}
	\end{longtable}
\end{center}
\begin{figure}[H] 
    \centering 
    \includegraphics[width=0.7\columnwidth]{immagini/casi_duso2.png}
    \caption{\label{fig:attori}Sotto casi d'uso visitatore avente appuntamento}
\end{figure}
\begin{center}
	\centering
	\renewcommand{\arraystretch}{1.5}
	\rowcolors{3}{tableLight}{}
	\begin{longtable}{  p{2.5cm} p{9.6cm} }
		\rowcolor{tableHead}
		\textbf{\textcolor{white}{Identificativo}} & \textbf{\textcolor{white}{Descrizione}} \\
		\endhead  
		
		UC6.1 &  \textit{Attori}: persona con o senza appuntamento \newline \textit{Scopo}: l'utente inserisce/dice il suo nome e cognome \newline \textit{Pre-condizione}: la Skill Concierge Croccante deve essere stata avviata \newline \textit{Post-condizione}: l'utente ha inserito/detto il suo nome e cognome \\
		
		UC6.2 &  \textit{Attori}: persona con o senza appuntamento \newline \textit{Scopo}: l'utente inserisce/dice il nome e cognome della persona che cerca \newline \textit{Pre-condizione}: la Skill Concierge Croccante deve essere stata avviata \newline \textit{Post-condizione}: l'utente ha inserito/detto il suo nome e cognome della persona cercata \\
		
		UC6.3 &  \textit{Attori}: persona con o senza appuntamento \newline \textit{Scopo}: l'utente inserisce/dice l'orario di appuntamento \newline \textit{Pre-condizione}: la Skill Concierge Croccante deve essere stata avviata \newline \textit{Post-condizione}: l'utente ha inserito/detto l'orario di appuntamento \\
		
		\rowcolor{white}
		\caption{Tabella sotto casi d'uso persona con appuntamento e non}
	\end{longtable}
\end{center}
\begin{figure}[H] 
    \centering 
    \includegraphics[width=0.7\columnwidth]{immagini/casi_duso21.png}
    \caption{\label{fig:attori}Sotto casi d'uso visitatore avente appuntamento}
\end{figure}
\begin{center}
	\centering
	\renewcommand{\arraystretch}{1.5}
	\rowcolors{3}{tableLight}{}
	\begin{longtable}{  p{2.5cm} p{9.8cm} }
		\rowcolor{tableHead}
		\textbf{\textcolor{white}{Identificativo}} & \textbf{\textcolor{white}{Descrizione}} \\
		\endhead  
		
		UC8.1 &  \textit{Attori}: persona \newline \textit{Scopo}: l'utente inserisce/dice il suo nome e cognome \newline \textit{Pre-condizione}: la Skill Concierge Croccante deve essere stata avviata \newline \textit{Post-condizione}: l'utente ha inserito/detto il suo nome e cognome \\
		
		UC8.2 &  \textit{Attori}: persona  \newline \textit{Scopo}: l'utente inserisce/dice il nome e cognome della persona che cerca \newline \textit{Pre-condizione}: la Skill Concierge Croccante deve essere stata avviata \newline \textit{Post-condizione}: l'utente ha inserito/detto il suo nome e cognome della persona cercata \\

		\rowcolor{white}
		\caption{Tabella sotto casi d'uso persona con appuntamento e non}
	\end{longtable}
\end{center}
\subsection{Casi d'uso - Postino/Corriere}
La seconda parte di analisi dei casi d'uso, è dedicata al servizio di consegna dei pacchi che viene svolto dal postino o dal corriere. In questa sezione si rappresenta graficamente i casi d'uso dell'utente identificato come "Corriere", che descrive le interazioni che l'attore svolge con il sistema.
\begin{figure}[H] 
    \centering 
    \includegraphics[width=1\columnwidth]{immagini/casi_duso3.png}
    \caption{\label{fig:attori}Casi d'uso corriere}
\end{figure}
Dall'elaborazione di tale analisi sono quindi emersi i seguenti casi d'uso riportati:
\begin{center}
	\centering
	\renewcommand{\arraystretch}{1.5}
	\rowcolors{3}{tableLight}{}
	\begin{longtable}{  p{2.5cm} p{9.8cm} }
		\rowcolor{tableHead}
		\textbf{\textcolor{white}{Identificativo}} & \textbf{\textcolor{white}{Descrizione}} \\
		\endhead  
		
		
		UC9 &  \textit{Attori}: corriere \newline \textit{Scopo}: l'utente può avviare Concierge Croccante \newline \textit{Pre-condizione}: il dispositivo Amazon deve aver avviato l'assistente Alexa \newline \textit{Post-condizione}: l'utente ha avviato Concierge Croccante \\
		
		UC10 &  \textit{Attori}: corriere \newline \textit{Scopo}: l'utente riceve un messaggio di benvenuto sul display del dispositivo \newline \textit{Pre-condizione}: la Skill Concierge Croccante deve essere stata avviata \newline \textit{Post-condizione}: l'utente riceve un messaggio di benvenuto sul display del dispositivo \\
		
		UC11 &  \textit{Attori}: corriere \newline \textit{Scopo}: l'utente riceve un messaggio vocale di benvenuto dalla Skill \mbox{Concierge} Croccante \newline \textit{Pre-condizione}: la Skill Concierge Croccante deve essere stata avviata \newline \textit{Post-condizione}: l'utente riceve un messaggio vocale dalla Skill C.C.\\
		
		UC12 &  \textit{Attori}: corriere \newline \textit{Scopo}: l'utente visualizza l'elenco sintetico ed essenziale di azioni sul \mbox{display} del dispositivo \newline \textit{Pre-condizione}: la Skill Concierge Croccante deve essere stata avviata \newline \textit{Post-condizione}: l'utente visualizza l'elenco di azioni sul display del dispositivo \\
		
		UC13 &  \textit{Attori}: corriere \newline \textit{Scopo}: viene esposto all'utente l'elenco sintetico ed essenziale di azioni \newline \textit{Pre-condizione}: la Skill Concierge Croccante deve essere stata avviata \newline \textit{Post-condizione}: viene esposto all'utente l'elenco di azioni disponibili da C.C. \\
		
		UC14 &  \textit{Attori}: corriere \newline \textit{Scopo}: l'utente può consegnare un pacco \newline \textit{Pre-condizione}: la Skill Concierge Croccante deve essere stata avviata \newline \textit{Post-condizione}: l'utente consegna il pacco \\
		
		UC15 &  \textit{Attori}: corriere \newline \textit{Scopo}: viene visualizzato un messaggio, su display e vocale, di errore specifico per l'eccezione riscontrata \newline \textit{Pre-condizione}: la Skill Concierge Croccante deve essere stata avviata \newline \textit{Post-condizione}: l'utente riceve un messaggio su display e vocale di errore\\
		
		UC17 &  \textit{Attori}: corriere \newline \textit{Scopo}: l'utente per consegnare il pacco richiede la presenza di una persona specifica per una firma \newline \textit{Pre-condizione}: la Skill Concierge Croccante deve essere stata avviata \newline \textit{Post-condizione}: l'utente riceve la firma desiderata e consegna il pacco\\
		
		UC18 &  \textit{Attori}: corriere \newline \textit{Scopo}: l'utente per consegnare il pacco richiede la presenza di una persona per una firma \newline \textit{Pre-condizione}: la Skill Concierge Croccante deve essere stata avviata \newline \textit{Post-condizione}: l'utente riceve la firma desiderata e consegna il pacco\\
		\rowcolor{white}
		\caption{Tabella casi d'uso corriere}
	\end{longtable}
\end{center}
\section{Voice User Interface - VUI}
\section{Graphical User Interface - GUI}
             
% !TEX encoding = UTF-8
% !TEX TS-program = pdflatex
% !TEX root = ../tesi.tex

%**************************************************************
\chapter{Progettazione e codifica}
\label{cap:progettazione}
In questo capitolo verrà tratta il secondo periodo del tirocinio: la progettazione e la codifica. Questa parte, avvenuta dopo l'Analisi dei Requisiti, è da ritenersi importante per progetto in quanto ha occupato gran parte del tempo lavorativo. Il primo passo è stato eseguire la progettata dell'architettura, in modo che i servizi necessari fossero correttamente configurati prima del loro effettivo utilizzo. Successivamente è stata eseguita la stesura del codice per la realizzazione della Skill. Nei paragrafi successivi seguiranno porzioni di codice significativo per la loro importanza e per il funzionamento di determinate funzionalità e/o servizi. Importante ricordare che il codice implementato e mostrato sarà in Node.js come quanto detto nel nel paragrafo \hyperref[nodejs]{1.4.1}. 
%**************************************************************
\section{Architettura}
Come riportato nel paragrafo \hyperref[serivizi_aws]{1.4.3}, il progetto si basa interamente sui servizi offerti dall'ecosistema Amazon. Di conseguenza è risultato facile ed immediato impostare i servizi dell'architettura in modo da rendere il tutto efficacie ed efficiente. In questo paragrafo si andrà a comprendere cosa offre ogni singolo servizio di AWS e come esso è stato impostato per il suo corretto funzionamento nella Skill.
\section{Configurazione servizi Amazon Web Service}
\subsection{AWS SES}
Amazon SES\footnote{\href{AWS SES. URL: https://aws.amazon.com/it/ses/}{https://aws.amazon.com/it/ses/}} (Simple Email Service) è un servizio di invio e-mail basato sul cloud messo a disposizione da Amazon per gli sviluppatori di applicazioni. Tale servizio è da considerarsi vantaggioso in quanto è affidabile e a costo ridotto, utile per qualunque tipo di azienda ed ideale per la realizzazione del progetto.
\newpage
\noindent Per utilizzare AWS SES è necessario impostare il servizio visitando la pagina\footnote{\href{AWS SES. URL: https://aws.amazon.com/it/ses/}{https://aws.amazon.com/it/ses/}} dedicata ed eseguire le istruzioni riportate:\\[0.5cm]
\begin{minipage}{0.47\textwidth}
	\begin{figure}[H]
		\includegraphics[width=6cm]{immagini/ses.png}
		\caption{\label{fig:icona_aws_ses}Icona AWS SES}
	\end{figure}
\end{minipage}
\begin{minipage}{0.5\textwidth}
	\begin{itemize}
		\item Eseguire l'accesso con le proprie credenziali di account AWS
    	\item Cliccare \texttt{Email Addresses }sul pannello a sinistra
    	\item Cliccare \texttt{Verify a New Email Address}
    	\item Inserire l'indirizzo e-mail da verificare che verrà utilizzato per inviare le notifiche
    	\item Confermare la verifica cliccando sul URL contenuto nella e-mail ricevuta.
	\end{itemize}
\end{minipage}
\\[0.5cm]
Il procedimento descritto sopra non fa altro che verificare l'indirizzo e-mail con il quale si andrà ad inviare messaggi di posta elettronica tramite la Skill. A questo punto è possibile mandare messaggi e notifiche tramite mail utilizzando l'indirizzo verificato prima.
\subsection{AWS S3}
\subsection{AWS IAM}
\subsubsection{AWS Cloud Watch}
\subsubsection{AWS DynamoDB}
\subsection{AWS Lambda ed Alexa Developer Console}

\section{Configurazione Google Calendar}

\section{Configurazione Slack}

\section{Codifica}
\subsection{Organizzazione del codice}
\subsection{Invio notifiche}
\subsubsection{Slack}
\subsubsection{E-mail}
\subsection{Lettura del calendario}
\subsection{Lettura del Database}
\subsection{Cambio di contesto}
             
% !TEX encoding = UTF-8
% !TEX TS-program = pdflatex
% !TEX root = ../tesi.tex

%**************************************************************
\chapter{Realizzazione e Testing}
\label{cap:realizzazione_testing}
%**************************************************************
             
% !TEX encoding = UTF-8
% !TEX TS-program = pdflatex
% !TEX root = ../tesi.tex

%**************************************************************
\chapter{Verifica e validazione}
\label{cap:verifica_validazione}
%**************************************************************
Ultima fase del periodo di tirocinio è stato svolto dal processo di verifica e validazione. La prima è un processo che si occupa di fornire evidenza oggettiva, ovvero che i risultati ottenuti come output dello sviluppo del software soddisfino i requisiti, la seconda invece è un processo per confermare in modo definitivo che le caratteristiche del software siano conformi ai bisogni dell'utente e all'uso previsto.
Per eseguire la verifica e la validazione è necessario l'esecuzione di test\footnote{In ambito informatico i test eseguiti nel processo di verifica sono eseguiti in maniera automatica durante lo sviluppo del software} che individuino carenze, correttezza e affidabilità. Nel progetto della Skill Concierge Croccante non è stato realizzato alcun test per il processo di verifica: viste le tempisti non è stato ritenuto necessario e si è pertanto scelto di eseguire test già esistenti nel servizio Amazon Lambda, sufficienti ad esaminare il codice e rilevare eventuali errori. Per quanto riguarda il processo di validazione\footnote{Attività di supporto la quale accerta che il prodotto dei processi rispetti le specifiche} è stato svolto, alla fine del periodo di tirocinio, un incontro  con il tutor aziendale e CFO Damiano Buscemi come conferma fiale la quale ha accertato le bontà richieste dal prodotto finale. Tale validazione non è stata altro che una dimostrazione funzionante della Skill, eseguita prima da me per spiegare spiegare le varie funzionalità, ed infine da Crispy Bacon come conferma.
\\[0.5cm]
\noindent Per quanto riguarda l'integrazione coi servizi utilizzati, nello specifico Google Calendar e Slack, sono stati creati dei file appositi da eseguire manualmente che verificano il corretto funzionamento di essi restituendo un risultato:
\begin{figure}[H]
	\includegraphics[width=13cm]{immagini/test-googleCalendar.png}
	\caption{\label{fig:test_googleCalendar}Esempio test funzionamento Google Calendar - Lista eventi}
\end{figure}
\newpage
\noindent L'immagine mostra l'esecuzione del file getListEventGoogleCallendar-WithOAuth.js adibito a verificare la validità del token OAuth per ottenere la lista di eventi della giornata corrente. Inoltre verifica la corretta chiamata dell'API \texttt{calendar.events.list}.
\begin{figure}[H]
	\includegraphics[width=13cm]{immagini/test-sendSlack.png}
	\caption{\label{fig:test_slack_send}Esempio test funzionamento Slack - Invio messaggio}
\end{figure}
\noindent L'immagine mostra l'esecuzione del file sendSlackMessage.js adibito a verificare la validità del token per inviare messaggi di notifica via Slack. Inoltre verifica la corretta chiamata dell'API \texttt{chat.postMessage}.
\begin{figure}[H]
	\includegraphics[width=13cm]{immagini/test-listSlack.png}
	\caption{\label{fig:test_slack_list}Esempio test funzionamento Slack - Lista membri}
\end{figure}
\noindent L'immagine mostra l'esecuzione del file getSlackListMemebers.js adibito a verificare la validità del token per ottenere la lista dei membri di Crispy Bacon. Inoltre verifica la corretta chiamata dell'API \texttt{users.list}.
             
% % !TEX encoding = UTF-8
% !TEX TS-program = pdflatex
% !TEX root = ../tesi.tex

%**************************************************************
\chapter{Valutazione finale}
\label{cap:valutazione_finale}
%**************************************************************
In questo ultimo capitolo verranno fatte le analisi dei risultati ottenuti, dei dati e delle informazioni raccolte ed emerse durante il tirocinio e una valutazione personale di come è stata vissuta tale esperienza. Verranno fatti anche dei commenti critici sul risultato del prodotto ottenuto, sul lavoro svolto e sugli strumenti utilizzati. Infine verranno fatte delle osservazioni sulla possibilità di miglioramenti sul prodotto, sulla metodologia di lavoro e possibilità di estensioni del codice.
\\[0.6cm]
Il progetto di stage aveva come obiettivo la realizzazione di una Skill, pensata per Amazon Echo Show, per l'accoglienza di visitatori utilizzando le funzionalità e i servizi offerti da Amazon Web Services.
Lo scopo principale del tirocinio era la sperimentazione con le tecnologie Node.js ed Alexa Presentation Language per realizzare un prototipo di prodotto su cui individuare le potenzialità e i limiti, confrontandole anche con tecnologie simili come Google Assistant. In tale contesto, ad inizio stage assieme al tutor aziendale sono stati definiti gli obiettivi minimi e massimi da portare a termine nell'arco delle 300-320 ore a disposizione. Gli obiettivi minimi racchiudono tutte le attività incentrate sullo studio ,sull'utilizzo delle tecnologie AWS e l'integrazione di servizi terzi finalizzati al mio stage. Gli obiettivi massimi, invece, riguardavano il soddisfacimento dei requisiti desiderabili quale la registrazione della durata degli appuntamenti determinati dall'inizio e dalla fine dell'accoglienza fisica fatta dal personale.
% - Risultato ottenuto
% - Analisi critica del prodotto e del lavoro di stage in generale
% - Il prodotto e utilizzato?
% - Valutazione degli strumenti utilizzati
% - Possibili punti di insoddisfazione e relativi miglioramenti
% - Possibili estensioni
\section{Risultati ottenuti}
A valle del lavoro svolto e dell'impegno dedicato, è possibile affermare che il risultato ottenuto dal progetto Concierge Croccante è da considerarsi soddisfacente rispetto alle attese. Questo motivato dal fatto che al termine del periodo di tirocinio gli obbiettivi fissati come obbligatori sono stati considerati completamente soddisfatti. In particolari, facendo riferimento all'identificazione dei requisiti riportata al punto \hyperref[indentificazione-requisiti]{2.3.1}, sono ritenuti soddisfatti i seguenti:
\newpage
\begin{center}
	\centering
	\renewcommand{\arraystretch}{1.5}
	\rowcolors{3}{tableLight}{}
	\begin{longtable}{  p{2.5cm} p{2.5cm} }
		\rowcolor{tableHead}
		\textbf{\textcolor{white}{Identificativo}} & \textbf{\textcolor{white}{Soddisfatto}} \\
		\endhead  
		RO1 & Si \\
		RO2 & Si \\ 
		RO3 & Si \\
		RO4 & Si \\
		RO5 & Si \\
		RO6 & Si \\
		RO7 & Si \\
		RO8 & Si \\
		RO9 & Si \\
		RO10 & Si \\
		RO11 & Si \\
	    RO12 & Si \\
	    RO13 & Si \\
	    RO14 & Si \\
	    RO15 & Si \\
	    RO16 & Si \\
	    RO17 & Si \\
		RO18 & Si \\
		RO19 & Si \\
		RD1 & Si \\
		RD2 & No \\
		RD3 & No \\
		RD4 & No \\
		\rowcolor{white}
		\caption{Tabella requisti soddisfatti}
	\end{longtable}
\end{center}
Inoltre, grazie ad una buona integrazione con il personale al di fuori del tutor aziendale, è stato possibile collaborare con il team UI/UX per una maggiore completezza e presentazione della Skill Concierge Croccante.
Un grande apprezzamento come risultato dello stage è aver conosciuto ed approfondito l'utilizzo delle API dei servizi Google Calendar e Slack integrati su ambienti cloud computing.

\section{Analisi critica del prodotto}
Analizzando criticamente il prodotto si può affermare che la Skill è stata sviluppata con una buona base conversazionale semplice e non eccessivamente lunga da svolgere, fattore rilevante visto il contesto degli assistenti vocali. Altro valore aggiunto è il fatto di aver realizzato un prodotto con funzionalità ben definite, così da non causare smarrimento all'utente durante l'utilizzo.
\\[0.5cm]
Osservando la tabella riportata sopra il requisito:\\
\begin{itemize}
    \item RD2 non è stato soddisfatto a causa del poco tempo rimasto a disposizione durante il periodo di progettazione e codifica. Era stato però pensato di implementare questa funzionalità utilizzando un secondo calendario dove inserire i periodi di assenza del dipendente;
    \item RD3 non è stato possibile completare in quanto non è stata trovata alcuna soluzione automatica in grado di soddisfare tale requisito senza obbligare l'utente ad indicare l'inizio dell'appuntamento alla Skill;
	\item RD4 non è stato possibile completare in quanto non è stata trovata alcuna soluzione automatica in grado di soddisfare tale requisito senza obbligare l'utente ad indicare il termine dell'appuntamento alla Skill.
\end{itemize}

\section{Analisi critica del lavoro}
Nel corso dello stage ho avuto la possibilità di mettere alla prova le mie competenze di organizzazione del lavoro e pianificazione delle attività per la prima volta al di fuori dell'ambito universitario, competenze acquisite per la maggior parte dal corso di Ingegneria del Software\footnote{SWE. URL: \href{https://www.math.unipd.it/~tullio/IS-1/2018/}{https://www.math.unipd.it/~tullio/IS-1/2018/}}. Nonostante io sia stato seguito ed aiutato dal mio tutor aziendale, la gestione temporale del progetto è stata per la maggior parte affidata a me. L'interazione con il tutor in questo contesto è stata principalmente espressa da incontri per ricevere feedback sull'organizzazione delle attività e sullo stato dei lavori. Questo mi ha permesso di maturare in maniera autonoma la priorità delle varie fasi, migliorando così le mie capacità organizzative grazie all'esperienza diretta. Altra soddisfazione ottenuta è data dall'esperienza acquisita utilizzando i servizi offerti dall'ecosistema Amazon Web Service con il quale è stato progettato il prodotto finale, molto apprezzato visto il numeroso ambito d'uso. Complessivamente sono molto soddisfatto dell'esperienza di stage svolta presso Crispy Bacon spa: oltre all'ottimo ambiente di lavoro e del personale, riscontrato in azienda, ho avuto modo di accrescere la mia figura professionale. Alla luce delle conoscenze e competenze acquisite durante il tirocinio posso affermare con sicurezza che le mie aspettative personali sono state completamente corrisposte.

\newpage
\section{Valutazione degli strumenti utilizzati}
Per quanto riguarda gli strumenti utilizzati ho potuto mettere in pratica, in maniera adattata alle esigenze dell'azienda, il modello di sviluppo agile Scrum, fin'ora solamente affrontato dal punto di vista teorico all'interno del corso di Ingegneria del Software. Mettere in pratica questo modello all'interno di un contesto aziendale mi ha permesso di comprenderne i benefici e gli svantaggi che ne consegue il suo utilizzo.\\
È stato inoltre molto apprezzato l'uso della suite di Amazon Web Sevice con il quale ho potuto mettermi alla prova integrando i principali servizi utilizzati come DynamoDB e SES. Il loro utilizzo infatti mi ha permesso di approfondire meglio le loro potenzialità, imparando così a comprendere l'ampio campo d'uso che queste offrono. In aggiunta utilizzando l'ambiente di sviluppo Visual Studio Code ho potuto procedere alla stesura del codice in maniera agevolata, questo grazie alla possibilità di estensione dell'IDE con plugin per supportare al meglio i linguaggi di programmazione.\\[0.6cm]

\noindent
Per quanto riguarda il linguaggio utilizzato ho avuto la possibilità di utilizzare e studiare alcune particolarità di Node.js utilizzando le funzioni Lambda per i sistemi serverless. Particolarità con le quali ho riscontrato più difficoltà nell'apprendimento è stata la comprensione delle chiamate asincrone con le keyword async/await.\\
Infine ho avuto modo di toccare con mano il concetto di API esplorandone sia la progettazione che l'implementazione. In questo caso sono entrato in contatto con l'aspetto pratico di nozioni che avevo precedentemente incontrato solo dal punto di vista teorico. Di conseguenza ho apprezzato l'esperienza nell'integrare servizi e tecnologie di vario tipo inglobandole in un unico applicativo da me creato.

\section{Possibili miglioramenti e conclusione}
Il progetto realizzato si presta sicuramente a dei miglioramenti sulle funzionalità realizzate, visto il tempo a disposizione non mi è stato possibile entrare in una fase di miglioramento dopo aver acquisito competenze di programmazione nuove in grado di perfezionare e arricchire il prodotto finale.\\
In conclusione l'esperienza che ho vissuto durante il periodo di stage è stata molto interessante e utile. Ho messo alla prova le mie competenze e le mie abilità, verificando che quanto appreso durante gli anni di studio è stato sufficiente ad affrontare un progetto di un certo calibro in quasi totale autonomia. Ovviamente è stato necessario dello studio personale sulle tecnologie non conosciute che sono state utilizzate, ma i corsi seguiti durante il triennio mi hanno fornito un bagaglio di conoscenze che ha facilitato questo studio.\\
Sono felice di avere svolto lo stage presso Crispy Bacon in quanto mi sono trovato in un ambiente molto positivo, stimolante e aperto alle novità. Altro fattore positivo dell'azienda è la giovane età che questa dimostra, manifestando interesse per le tecnologie nuove e quelle in evoluzione, seguendo le tendenze del mercato e stando così al passo coi tempi.             % Product Design Freeze e SOP
% \input{capitoli/capitolo-7}             % Conclusioni
\appendix                               
\input{capitoli/capitolo-A}             % Appendice A

%**************************************************************
% Materiale finale
%**************************************************************
\backmatter
\printglossaries
\input{inizio-fine/bibliografia}
\end{document}
